\section{Modello XY}

%-----------------------------------------%
%				Prima slide				  %
%	   Caratterizzazione del modello   	  %
%-----------------------------------------%
\begin{frame}
    \frametitle{Caratterizzazione}
    \framesubtitle{}

    \begin{columns}
        \begin{column}{0.33\textwidth}
            \begin{block}{Termalizzazione}
                \begin{itemize}[itemsep=0.5em, label=$\diamond$]
                    \item $t_{ter}$ maggiori per $T \to 0$
                    \item $t_{ter}^{max}\,\simeq\,300$ sweeps
                \end{itemize}

                \vspace{0.5cm}

                \centering
                \includegraphics[width=\textwidth]{Immagini/simXY/term_200_1.0.pdf}
            \end{block}
        \end{column}
    
        \begin{column}{0.33\textwidth}
            \begin{block}{Auto-correlazione}
                \begin{itemize}[itemsep=0.5em, label=$\diamond$]
                    \item $t_{c}$ maggiori per $T \to 0$
                    \item $t_{c}^{max}\,\simeq\,5000$ sweeps
                \end{itemize}

                \vspace{0.5cm}

                \centering
                \includegraphics[width=\textwidth]{Immagini/simXY/auto_50_1.0.pdf}
            
            \end{block}
        \end{column}

        \begin{column}{0.33\textwidth}
            \begin{block}{Blocchi}
                \begin{itemize}[itemsep=0.5em, label=$\diamond$]
                    \item $l_{blk}$ maggiori per $T \to 0$
                    \item $l_{blk}^{max}\,\simeq\,7000-1000$ sweeps
                \end{itemize}

                \vspace{0.5cm}

                \centering
                \includegraphics[width=\textwidth]{Immagini/simXY/lblk_100_0.5.pdf}

            \end{block}        
        \end{column}
    \end{columns}
\end{frame}



%-----------------------------------------%
%			  Seconda slide				  %
%	  Studio configurazioni low-highT  	  %
%-----------------------------------------%
\begin{frame}
    \frametitle{Configurazioni}
    \framesubtitle{}

    \begin{columns}
        \begin{column}{0.5\textwidth}
            \begin{block}{Bassa temperatura}

            \centering
            \includegraphics[width=0.8\textwidth]{Immagini/simXY/conf_T0.5.png}

            \end{block}
        \end{column}
    
        \begin{column}{0.5\textwidth}
            \begin{block}{Alta temperatura}

                \centering
                \includegraphics[width=0.8\textwidth]{Immagini/simXY/conf_T2.5.png}

            \end{block}
        \end{column}
    \end{columns}

\end{frame}



%-----------------------------------------%
%			   Terza slide				  %
%	       Esempio di vortice        	  %
%-----------------------------------------%
\begin{frame}
    \frametitle{Vortice}
    \framesubtitle{}

    \begin{columns}
        \begin{column}{0.5\textwidth}
            \centering
            \includegraphics[width=0.8\textwidth]{Immagini/simXY/conf_vortice.png}
        \end{column}
    
        \begin{column}{0.5\textwidth}
            \begin{itemize}[itemsep=0.5em, label=$\diamond$]
                \item Winding number $w\,=\,1$
                \item Transizione di Kosterlitz-Thouless
            \end{itemize}
        \vspace{12pt}

        \begin{block}{Temperatura critica}
            \centering
            $T_c\,=\,\frac{\pi J}{2} $
        \end{block}
        \end{column}
    \end{columns}

\end{frame}



%-----------------------------------------%
%			   Quarta slide				  %
%	   Energia e Magnetizzazione XY  	  %
%-----------------------------------------%
\begin{frame}
    \frametitle{Osservabili}
    \framesubtitle{}

    \begin{columns}
        \begin{column}{0.5\textwidth}
            \begin{block}{Energia}
                \centering
                \includegraphics[width=\textwidth]{Immagini/simXY/eneXY.png}
            \end{block}
        \end{column}
    
        \begin{column}{0.5\textwidth}
            \begin{block}{Applicazione campo}
                \centering
                \includegraphics[width=\textwidth]{Immagini/simXY/magnXY.png}
            \end{block}
        \end{column}
    \end{columns}

\end{frame}
