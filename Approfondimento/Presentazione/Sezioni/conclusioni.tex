\section{Conclusioni}

%----------------------------------------%
%		      Prima slide	     	     %
%	   Riassunto studio effettuato       %
%----------------------------------------%
\begin{frame}
    \frametitle{Riassunto}
    \framesubtitle{}

    \begin{columns}
        \begin{column}{0.33\textwidth}
            \begin{block}{Ising 1D}

                \begin{itemize}[itemsep=0.5em, label=$\diamond$]
                    \item no fase ordinata con $h = 0.0$
                    \item approccio MF fisicamente errato
                    \item verifica criterio di Ginzburg
                \end{itemize}
            
            \end{block}
        \end{column}
    
        \begin{column}{0.33\textwidth}
            \begin{block}{Ising 2D}

                \begin{itemize}[itemsep=0.5em, label=$\diamond$]
                    \item transizione di fase a $T_c \neq 0$
                    \item modello ottimo per lo studio del punto critico
                    \item rottura di simmetria discreta
                \end{itemize}
            
            \end{block}
        \end{column}

        \begin{column}{0.33\textwidth}
            \begin{block}{Modello XY}

                \begin{itemize}[itemsep=0.5em, label=$\diamond$]
                    \item teorema di Mermin-Wagner
                    \item transizione di fase topologica
                    \item quasi long-range order per $T\,<\,T_{KT}$
                \end{itemize}

            \end{block}        
        \end{column}
    \end{columns}

\end{frame}



%-----------------------------------------%
%		   	  Seconda slide				  %
%	    Ringraziamento a tutti per   	  %
%-----------------------------------------%
\begin{frame}
    \frametitle{Fine}
    \framesubtitle{}

    \begin{center}
        Grazie per l'attenzione
    \end{center}
    
\end{frame}