\section{Introduzione}

%-----------------------------------------%
%				Prima slide				  %
%	Hamiltoniana di Ising e spiegazione	  %
%-----------------------------------------%
\begin{frame}
    \frametitle{Hamiltoniana}
    \framesubtitle{}

    \begin{columns}
        \begin{column}{0.5\textwidth}
			
			\vspace{12pt}

			\begin{equation*}
				H\,=\,-J\sum_{\left<i,j\right>}\sigma_i\sigma_j\,-\,h\sum_{i} \sigma_i
				\label{eq: Ham_Ising}
			\end{equation*}

			\vspace{12pt}

            \begin{itemize}[itemsep=0.5em, label=$\diamond$]
                \item Interazione fra primi vicini
                \item Accoppiamento con un campo esterno
            \end{itemize}

        \end{column}

        
        \begin{column}{0.5\textwidth}
				\centering
				\includegraphics[width=\textwidth]{Immagini/Introduzione/modelloIsing1D_pbc.png}
				{\scriptsize Modello di Ising 1D con condizioni periodiche.}
		
		\end{column}
      \end{columns}
  
\end{frame}



%----------------------------------------------%
%			  Seconda slide				       %
%	Modello di Ising 1D e soluzione analitica  %
%----------------------------------------------%
\begin{frame}
    \frametitle{Modello di Ising 1D}
    \framesubtitle{}

    \begin{columns}

        \begin{column}{0.5\textwidth}
			\begin{block}{Campo medio}     

                \begin{itemize}[itemsep=0.5em, label=$\diamond$]
                    \item Transizione di fase a $T_c \neq 0$
                \end{itemize}
                
			    $$ m\,=\,\tanh{\left[\beta\left(h\,+\,Jn_{nn}m\right)\right]} $$

			    \vspace{0.7cm}

                \centering
				\includegraphics[width=0.8\textwidth]{Immagini/Introduzione/magn_Ising1D_mf.png}

            \end{block}
        \end{column}


        \begin{column}{0.5\textwidth}
            \begin{block}{Soluzione analitica}
			
			    \begin{itemize}[itemsep=0.5em, label=$\diamond$]
                    \item No magnetizzazione se $T\,\neq\,0$, $h\,=\,0$
                \end{itemize}

			    \begin{equation*}
			    	m\,=\,\frac{\sinh{\left(\beta h\right)}}{\sqrt{e^{-4\beta J}\,+\,\sinh^2{\left(\beta h\right)}}}
			    	\label{eq: magn_Ising1D_AS}
			    \end{equation*}

			    \vspace{0.1cm}

                \centering
				\includegraphics[width=0.8\textwidth]{Immagini/Introduzione/magn_Ising1D_sa.png}
            
            \end{block}
        \end{column}
      \end{columns}
  
\end{frame}



%----------------------------------------------%
%			     Terza slide			       %
%	Modello di Ising 2D e transizione di fase  %
%----------------------------------------------%
\begin{frame}
    \frametitle{Modello di Ising 2D}
    \framesubtitle{}

    \begin{columns}

        \begin{column}{0.55\textwidth}

			\begin{itemize}[itemsep=0.5em, label=$\diamond$]
                \item Soluzione analitica per $h = 0$
                \item Sistema presenta una transizione di fase a $T_c \neq 0$
            \end{itemize}

			\begin{equation*}
				m\left(\beta,\,h=0\right)\,=\,
				\begin{cases}
				\left[1\,-\,\dfrac{1}{\sinh^4{\left(2\beta J\right)}}\right]^{\frac{1}{8}}\,\, T\,<\,T_c \\
				0 \qquad \qquad \qquad \qquad \,\,\,\, T\,>\,T_c
				\end{cases}
				\label{eq: magn_Ising2D_AS}
			\end{equation*}

        \end{column}
        
        \vspace{0.3cm}

        \begin{column}{0.45\textwidth}
			
			\centering
            \includegraphics[width=\textwidth]{Immagini/Introduzione/magn_Ising2D.png}

            \begin{tabular}{cc}
                \includegraphics[width=0.45\textwidth]{Immagini/Introduzione/cg_1000_1.0.png} &
                \includegraphics[width=0.45\textwidth]{Immagini/Introduzione/cg_1000_3.0.png} \\
            \end{tabular}
        
        \end{column}
      \end{columns}
  
\end{frame}



%----------------------------------------------%
%			     Quarta slide			       %
%	Modello XY e diffetti parametro d'ordine   %
%----------------------------------------------%
\begin{frame}
    \frametitle{Modello XY}
    \framesubtitle{}

    \begin{columns}

        \begin{column}{0.55\textwidth}

            \begin{block}{Hamiltoniana}
			    \begin{equation*}
                    H\,=\,-J\sum_{\left<i,j\right>}\vec{s_i} \cdot \vec{s_j}\,-\,\sum_i \vec{h} \cdot \vec{s_i} 
				    \label{eq: magn_Ising2D_AS}
			    \end{equation*}
            \end{block}

            \vspace{0.5cm}

			\begin{itemize}[itemsep=0.5em, label=$\diamond$]
                \item Simmetria continua
                \item Difetti topologici (vortici)
            \end{itemize}

        \end{column}
        
        \vspace{0.3cm}

        \begin{column}{0.45\textwidth}
			
			\centering
            \includegraphics[width=\textwidth]{Immagini/Introduzione/modelloXY.png}
        
        \end{column}
      \end{columns}
  
\end{frame}
