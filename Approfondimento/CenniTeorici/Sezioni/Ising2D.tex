\section{Modello di Ising 2D}

Il modello di Ising 2D consiste in un reticolo quadrato bidimensionale di spin, i quali possono assumere solamente i valori 
$\pm 1$. Nel limite termodinamico, tale modello presenta una transizione di fase in prossimità di una temperatura critica Tc.
Per $T\,>\,Tc$ il sistema si comporta in modo paramagnetico, altrimenti si ha una magnetizzazione spontanea, evidenza 
di ordine nel reticolo di spin. Il modello presenta soluzione analitica solamente con campo magnetico nullo $h\,=\,0$, 
determinata da Onsager nel 1944. La magnetizzazione 

\begin{equation}
    m\left(\beta,\,h=0\right)\,=\,
    \begin{cases}
    \left[1\,-\,\dfrac{1}{\sinh^4{\left(2\beta J\right)}}\right]^{\frac{1}{8}} \qquad \qquad T\,<\,T_c \\
    0 \qquad \qquad \qquad \qquad \qquad \qquad \,\,\,\, T\,>\,T_c
    \end{cases}
    \label{eq: magn_Onsager_1944}
\end{equation}
presenta una discontinuità alla temperatura critica 

\begin{equation}
    T_c\,=\,\frac{2J}{\ln{\left(1\,+\,\sqrt{2}\right)}}\,,
    \label{eq: tc_Ising2D_Ons}
\end{equation}
evidenza ulteriore di una transizione di fase.


\subsection{Domain walls}


Consideriamo ora un domain walls in $d$ dimensioni in un sistema di dimensione lineare $La$, dove $L$ è un numero reale ed $a$ invece 
è il passo reticolare. In analogia con quanto osservato per il modello di Ising 1D, la variazione di energia legata a questo difetto 
nell'orientamento del sistema è pari a 

\begin{equation}
    \Delta E\,=\,2JL^{d-1}
    \label{eq: ene_dw_IsingdD}
\end{equation}

L'entropia del domain wall è legata al numero di modi in cui si può costruire tale interfaccia. Per un singolo domain wall si può 
stimare che 

\begin{equation}
    S \gtrsim k_B \ln{\left(L\right)}
    \label{eq: entr_dw_IsingdD}
\end{equation}

L'energia libera associata alla presenza dell'interfaccia è pari a 

\begin{equation}
    A \simeq 2JL^{d-1}\,-\,k_B T\ln{\left(L\right)},
    \label{eq: freeE_dw_IsingdD}
\end{equation}

che è dominata dal termine energetico per ogni dimensione del reticolo $d\,\geq 2$, poichè nel limite termodinamico il termine 
logaritmico è trascurabile. Per provare l'esistenza di ferromagnetismo in modo più quantitativo, basta mostrare che il valor medio 
dello spin sia diverso da zero. Nel caso del modello di Ising 2D è possibile mostrare che se gli spin che fanno parte della cornice 
esterna del retico sono positivi, la probabilità di avere uno spin negativo nel centro del sistema è pari a 

\begin{equation}
    p_{-}\,<\,\frac{1}{2}\frac{\exp{\left(-2\beta J\right)}}{4 \left(1\,-\,3\exp{\left(-2\beta J\right)}\right)^4}.
    \label{eq: probm_Ising2D}
\end{equation}

Il secondo membro della relazione \eqref{eq: probm_Ising2D} può essere reso minore di $1/2$ scegliendo una temperatura opportuna ed 
in modo indipendente dalla dimensione del reticolo (ossia del parametro $N$ introdotto in precedenza). Questo evidenzia come possa 
presentarsi long range order e di conseguenza magnetizzazione finita per $0\,\leq T\,<\,T_c$. Per sottolineare ulteriormente le 
differenze fra il modello di Ising 1D e quello bi-dimensionale consideriamo ora il mapping riportato in Figura \ref{fig: map_2to1_Ising}, 
in cui i siti reticolari di un modello bi-dimensione vengono mappati su una catena lineare (mantendo l'interazione con i primi 
vicini di partenza).

\begin{figure}[H]
    \centering
    \includegraphics[width=0.5\textwidth]{Immagini/map_2to1_Ising.png}
    \caption{Mapping di un modello di Ising 2D su un modello di Ising 1D. Immagine da \cite{galliFSA}.}
    \label{fig: map_2to1_Ising}
\end{figure}

Sebbene si possa mappare il reticolo quadrato in una catena di spin, il motivo per cui tale reticolo lineare è ordinato a temperatura 
finita è da ricercare nel range dell'interazione. Nel caso del modello di Ising 1D con interazione fra primi vicini, la stessa è short 
range, poichè coinvolge solamente i siti adiacenti a quello preso in considerazione. Nel caso invece di catena di spin ottenuta come 
risultato del mapping di un reticolo quadrato in uno lineare, l'interazione è long-range, ed a ogni nuovo cambio di direzione delle 
spirali concentriche con cui si visitano tutti i siti reticolari tale lunghezza d'interazione aumenta. Nel caso della catena di 
spin ottenuta a partire da un reticolo quadrato anche i domain walls interagiscono fra loro con un potenziale di tipo long-range, andando 
ad invalidare il discorso fatto in precedenza e rendendo possibile una magnetizzazione non nulla anche per una catena di spin. 



\subsection{Fenomeni ad invarianza di scala}



