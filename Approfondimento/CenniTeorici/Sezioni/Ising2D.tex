\section{Modello di Ising 2D}

Il modello di Ising 2D consiste in un reticolo quadrato bidimensionale di spin, i quali possono assumere solamente i valori 
$\pm 1$. Nel limite termodinamico, tale modello presenta una transizione di fase in prossimità di una temperatura critica Tc.
Per $T\,>\,Tc$ il sistema si comporta in modo paramagnetico, altrimenti si ha una magnetizzazione spontanea, evidenza 
di ordine nel reticolo di spin. Il modello presenta soluzione analitica solamente per il caso con campo magnetico $h\,=\,0$, 
determinata da Onsager nel 1944. La magnetizzazione 

\begin{equation}
    m\left(\beta,\,h=0\right)\,=\,
    \begin{cases}
    \left[1\,-\,\dfrac{1}{\sinh^4{\left(2\beta J\right)}}\right]^{\frac{1}{8}} \qquad \qquad T\,<\,T_c \\
    0 \qquad \qquad \qquad \qquad \qquad \qquad \,\,\,\, T\,>\,T_c
    \end{cases}
    \label{eq: magn_Onsager_1944}
\end{equation}
presenta una discontinuità alla temperatura critica 

\begin{equation}
    T_c\,=\,\frac{2J}{\ln{\left(1\,+\,\sqrt{2}\right)}}\,,
    \label{eq: tc_Ising2D_Ons}
\end{equation}
evidenza ulteriore di una transizione di fase. L'energia interna per unità di spin è data da 

\begin{equation}
    u_l(0, T) = -2J \tanh(2\beta J) + \frac{k}{2\pi} \frac{dk}{d\beta} \int_{0}^{\pi} 
    \frac{\sin^2(\phi)}{\Delta (1 + \Delta)} d\phi
    \label{eq: ene_Onsager_1944}
\end{equation}
dove

\begin{equation}
    k\,=\,\frac{2 \sinh{\left(2 \beta J\right)}}{\cosh^2{\left(2\beta J\right)}}
    \label{eq: k_Ising2D_Ons}
\end{equation}

\begin{equation}
    \Delta = \sqrt{1 - k^2 \sin^2(\phi)}.
    \label{eq: Delta_Ising2D_Ons}
\end{equation}



\subsection{Domain walls}

\subsection{Scale free fenomena}


