Il modello di Ising consiste in un reticolo che presenta un momento magnetico (o spin) in ogni sito. Nel 
modello questi spin assumono la forma più semplice possibile, non particolarmente realistica, di variabili 
scalari $\sigma_i$ di valori $\pm 1$, rappresentanti rispettivamente dipoli unitari rivolti verso l'alto oppure 
verso il basso. Tali spin interagiscono fra loro e possono accoppiarsi ad un campo magnetico esterno e 
per tale motivo l'Hamiltoniana del sistema assume la forma 

\begin{equation}
    H\,=\,-J\sum_{\left<ij\right>} \sigma_i \sigma_j\,-\,h\sum_{i} \sigma_i,
    \label{eq: ising_ham}
\end{equation}

dove la notazione $\left<ij\right>$ denota una somma su primi vicini. Se il parametro $J$ è positivo i dipoli vicini 
tendono ad allinearsi e quindi il modello è di tipo ferromagnetico, altrimenti quando $J\,<\,0$ si ha anti-allineamento 
e fenomenologia anti-ferromagnetica.