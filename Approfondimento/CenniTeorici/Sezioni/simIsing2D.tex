\section{Simulazioni modello di Ising 2D}

Per il modello di Ising 2D ho lavorato sia con l'algoritmo di Metropolis che con quello di Wolff, nettamente più efficace rispetto al 
precedente nella caratterizzazione del punto critico. In entrambi i casi, la prima fase dello studio del modello di Ising 2D è incentrata 
sulla determinazione dei parametri ottimali per ottenere dei valori d'aspettazione statisticamente rilevanti.
In questa fase preliminare ho lavorato con cinque dimensioni reticolari $N\,\in\,\left\{100,\,200,\,300,\,400,\,500\right\}$ e 
cinque temperature $T\,\in\,\left\{1.0,\,1.5,\,2.0,\,2.5,\,3.0,\,3.5\right\}$. Per ognuna di queste coppie dimensione-temperatura 
ho considerato quattro seed differenti del generatore di numeri casuali in modo da poter analizzare il comportamento del sistema 
lungo differenti traiettorie nello spazio delle fasi. La seconda fase consiste invece nel calcolo dei valori d'aspettazione, da confrontare 
poi con quanto noto in letteratura.







Il sistema viene inizializzato con tutti gli spin orientati up (ossia con valore pari ad 1), poichè questo consente di avere delle 
termalizzazioni di durata inferiore. Come è possibile osservare nelle figure riportate in seguito, la fase di transitorio è praticamente 
istantanea per temperature minori della temperatura critica o nettamente superiori. Nell'intorno di $T_c$ si osservano termalizzazioni 
più lunghe, che risultano essere di circa 500 mosse Monte-Carlo per $T\,=\,2.5$. Sebbene l'algoritmo di Metropolis non sia adeguato 
per caratterizzare il punto critico, potrebbe essere comunque interessante valutare cosa accada per $T\,\in\,\left\{2.1,\,2.2,\,2.3,\,2.4\right\}$, 
perchè il confronto con i risultati prodotti con l'algoritmo di Wolff sarebbe comunque significativo. Per questo motivo, oltre allo 
studio della termalizzazione sul range di temperature "ampio" riportato in precedenza, ho effettuato uno studio più focalizzato sull'
intorno del punto critico, osservando come per $T\,\to\,T_c^+$ la durata della fase di transitorio risulti essere nettamente maggiore, 
richiedendo oltre 10000 mosse Monte-Carlo. Inoltre, nell'intorno del punto critico, le fluttuazioni nella magnetizzazione, parametro 
d'ordine per il modello di Ising, persistono per un numero considerevole di mosse MC, il che implica con ogni probabilità che saranno 
necessari dei blocchi di dimensioni maggiori nell'intorno del punto critico. Le simulazioni successive sono state svolte con 500 mosse 
Monte-Carlo di termalizzazione, tranne quelle nell'intorno di $T_c$ le quali hanno richiesto una termalizzaione di 10000 mosse MC.
