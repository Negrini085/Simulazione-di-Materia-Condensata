\documentclass[a4paper,11pt]{article}

% Pacchetti di base
\usepackage[utf8]{inputenc} % Per caratteri UTF-8
\usepackage[T1]{fontenc} % Font migliorati
\usepackage[italian]{babel} % Lingua italiana
\usepackage{amsmath, amssymb, amsthm} % Matematica avanzata
\usepackage{geometry} % Gestione margini
\geometry{a4paper, margin=2.5cm}
\usepackage{graphicx} % Per immagini
\usepackage{hyperref} % Collegamenti ipertestuali
\usepackage{cleveref} % Riferimenti migliorati
\usepackage{fancyhdr} % Intestazioni e piè di pagina
\usepackage{enumitem} % Liste personalizzate
\usepackage{xcolor} % Colori
\usepackage{booktabs} % Tabelle eleganti
\usepackage{float}

% Impostazioni per link ipertestuali
\hypersetup{
    colorlinks=true,
    linkcolor=blue,
    citecolor=red,
    urlcolor=blue,
    pdftitle={Dispense Universitarie},
    pdfauthor={Autore},
    pdfsubject={Teoria},
    pdfkeywords={LaTeX, Dispense, Università}
}

% Ambiente per teoremi, definizioni e dimostrazioni
\theoremstyle{definition}
\newtheorem{definition}{Definizione}[section]
\theoremstyle{plain}
\newtheorem{theorem}{Teorema}[section]
\newtheorem{corollary}{Corollario}[theorem]
\theoremstyle{remark}
\newtheorem{remark}{Osservazione}[section]

% Numerazione delle equazioni
\numberwithin{equation}{section}

% Intestazione e piè di pagina
\pagestyle{fancy}
\fancyhf{}
\fancyhead[L]{\textit{Dispense Universitarie}}
\fancyhead[R]{\textit{\leftmark}}
\fancyfoot[C]{\thepage}

% Titolo e autore
\title{\textbf{Il modello di Ising}}
\author{Autore: Filippo Negrini \\
        Corso: Simulazione di Materia Condensata e Biosistemi \\
        Università: Università degli Studi di Milano}
\date{\today}

\begin{document}

% Pagina del titolo
\maketitle
\tableofcontents
\newpage

\section{Introduzione}

%-----------------------------------------%
%				Prima slide				  %
%	Hamiltoniana di Ising e spiegazione	  %
%-----------------------------------------%
\begin{frame}
    \frametitle{Hamiltoniana}
    \framesubtitle{}

    \begin{columns}
        \begin{column}{0.5\textwidth}
			
			\vspace{12pt}

			\begin{equation*}
				H\,=\,-J\sum_{\left<i,j\right>}\sigma_i\sigma_j\,-\,h\sum_{i} \sigma_i
				\label{eq: Ham_Ising}
			\end{equation*}

			\vspace{12pt}

            \begin{itemize}[itemsep=0.5em, label=$\diamond$]
                \item Interazione fra primi vicini
                \item Accoppiamento con un campo esterno
            \end{itemize}

        \end{column}

        
        \begin{column}{0.5\textwidth}
				\centering
				\includegraphics[width=\textwidth]{Immagini/Introduzione/modelloIsing1D_pbc.png}
				{\scriptsize Modello di Ising 1D con condizioni periodiche.}
		
		\end{column}
      \end{columns}
  
\end{frame}



%----------------------------------------------%
%			  Seconda slide				       %
%	Modello di Ising 1D e soluzione analitica  %
%----------------------------------------------%
\begin{frame}
    \frametitle{Modello di Ising 1D}
    \framesubtitle{}

    \begin{columns}

        \begin{column}{0.5\textwidth}
			\begin{block}{Campo medio}     

                \begin{itemize}[itemsep=0.5em, label=$\diamond$]
                    \item Transizione di fase a $T_c \neq 0$
                \end{itemize}
                
			    $$ m\,=\,\tanh{\left[\beta\left(h\,+\,Jn_{nn}m\right)\right]} $$

			    \vspace{0.7cm}

                \centering
				\includegraphics[width=0.8\textwidth]{Immagini/Introduzione/magn_Ising1D_mf.png}

            \end{block}
        \end{column}


        \begin{column}{0.5\textwidth}
            \begin{block}{Soluzione analitica}
			
			    \begin{itemize}[itemsep=0.5em, label=$\diamond$]
                    \item No magnetizzazione se $T\,\neq\,0$, $h\,=\,0$
                \end{itemize}

			    \begin{equation*}
			    	m\,=\,\frac{\sinh{\left(\beta h\right)}}{\sqrt{e^{-4\beta J}\,+\,\sinh^2{\left(\beta h\right)}}}
			    	\label{eq: magn_Ising1D_AS}
			    \end{equation*}

			    \vspace{0.1cm}

                \centering
				\includegraphics[width=0.8\textwidth]{Immagini/Introduzione/magn_Ising1D_sa.png}
            
            \end{block}
        \end{column}
      \end{columns}
  
\end{frame}



%----------------------------------------------%
%			     Terza slide			       %
%	Modello di Ising 2D e transizione di fase  %
%----------------------------------------------%
\begin{frame}
    \frametitle{Modello di Ising 2D}
    \framesubtitle{}

    \begin{columns}

        \begin{column}{0.55\textwidth}

			\begin{itemize}[itemsep=0.5em, label=$\diamond$]
                \item Soluzione analitica per $h = 0$
                \item Sistema presenta una transizione di fase a $T_c \neq 0$
            \end{itemize}

			\begin{equation*}
				m\left(\beta,\,h=0\right)\,=\,
				\begin{cases}
				\left[1\,-\,\dfrac{1}{\sinh^4{\left(2\beta J\right)}}\right]^{\frac{1}{8}}\,\, T\,<\,T_c \\
				0 \qquad \qquad \qquad \qquad \,\,\,\, T\,>\,T_c
				\end{cases}
				\label{eq: magn_Ising2D_AS}
			\end{equation*}

        \end{column}
        
        \vspace{0.3cm}

        \begin{column}{0.45\textwidth}
			
			\centering
            \includegraphics[width=\textwidth]{Immagini/Introduzione/magn_Ising2D.png}

            \begin{tabular}{cc}
                \includegraphics[width=0.45\textwidth]{Immagini/Introduzione/cg_1000_1.0.png} &
                \includegraphics[width=0.45\textwidth]{Immagini/Introduzione/cg_1000_3.0.png} \\
            \end{tabular}
        
        \end{column}
      \end{columns}
  
\end{frame}



%----------------------------------------------%
%			     Quarta slide			       %
%	Modello XY e diffetti parametro d'ordine   %
%----------------------------------------------%
\begin{frame}
    \frametitle{Modello XY}
    \framesubtitle{}

    \begin{columns}

        \begin{column}{0.55\textwidth}

            \begin{block}{Hamiltoniana}
			    \begin{equation*}
                    H\,=\,-J\sum_{\left<i,j\right>}\vec{s_i} \cdot \vec{s_j}\,-\,\sum_i \vec{h} \cdot \vec{s_i} 
				    \label{eq: magn_Ising2D_AS}
			    \end{equation*}
            \end{block}

            \vspace{0.5cm}

			\begin{itemize}[itemsep=0.5em, label=$\diamond$]
                \item Simmetria continua
                \item Difetti topologici (vortici)
            \end{itemize}

        \end{column}
        
        \vspace{0.3cm}

        \begin{column}{0.45\textwidth}
			
			\centering
            \includegraphics[width=\textwidth]{Immagini/Introduzione/modelloXY.png}
        
        \end{column}
      \end{columns}
  
\end{frame}

\section{Modello di Ising 1D}

Il modello di Ising 1D è uno dei pochi modelli della meccanica statistica che presenta una soluzione esatta.
Il reticolo che prendiamo in considerazione in questo caso è lineare, tale per cui ogni sito reticolare presenta 
solo due primi vicini. Lavorando con condizioni periodiche al contorno, l'N-esimo spin diventa un vicino del 
primo ed il sistema si chiude ad anello, come è possibile apprezzare in Figura \ref{fig: Ising1D_pbc}.

\begin{figure}[h!]
    \centering
    \includegraphics[width=0.7\textwidth]{Immagini/Ising1D_pbc.png}
    \caption{L'immagine (a) è un esempio di modello di Ising 1D senza pbc, mentre in (b) si può apprezzare 
    come la catena si chiuda su se stessa nel caso di condizioni periodiche al contorno. }
    \label{fig: Ising1D_pbc}
\end{figure}


\subsection{Soluzione esatta}

Considerare un sistema con condizioni periodiche al contorno, come (b) in Figura \ref{fig: Ising1D_pbc}, consente 
di scrivere l'Hamiltoniana in forma simmetrica come 

\begin{equation}
    H\,=\,-J\sum_{i} \sigma_i \sigma_{i+1}\,-\,\frac{h}{2}\sum_{i} \left(\sigma_i\,+\,\sigma_{i+1}\right),
    \label{eq: ising_ham_sim}
\end{equation}

dato che $\sigma_{N+1}\,=\,\sigma_1$. La funzione di partizione del sistema è data dalla somma su tutte le possibili 
configurazioni del sistema, che si traduce in 

\begin{equation}
    Q\left(h,\,T\right)\,=\,\sum_{\sigma_1=\pm 1} \cdots \sum_{\sigma_N=\pm 1} \exp{\left\{\beta\left[J\sum_i \sigma_i \sigma_{i+1}\,+\,\frac{h}{2}\sum_i \left(\sigma_i\,+\,\sigma_{i+1}\right)\right]\right\}}
    \label{eq: part_func}
\end{equation}

Definendo una matrice P come

\begin{equation}
    P = \begin{pmatrix}
    e^{\beta\left(J\,+\,h\right)} & e^{-\beta J} \\\\
    e^{-\beta J} & e^{\beta\left(J\,-\,h\right)}
    \end{pmatrix}
    \label{eq: mat_P}
\end{equation}

è possibile riscrivere la funzione di partizione in termini matriciali

\begin{equation}
    Q\left(h,\,T\right)\,=\,\sum_{\sigma_1=\pm 1} \cdots \sum_{\sigma_N=\pm 1} \langle \sigma_1 | P | \sigma_2 \rangle \langle \sigma_2 | P | \sigma_3 \rangle \cdots \langle \sigma_{N-1} | P | \sigma_N \rangle \langle \sigma_N | P | \sigma_1 \rangle 
    \label{eq: part_func_mat}
\end{equation}

Notando che sono presenti $N\,-\,1$ completezze, è possibile procedere ad una semplificazione estrema della relazione 
\eqref{eq: part_func_mat} che consente di apprezzare come la funzione di partizione altro non sia che la traccia della matrice P 
elevata alla N. 

\begin{equation}
    Q\left(h,\,T\right)\,=\,\sum_{\sigma_1=\pm 1} \langle \sigma_1 | P^N | \sigma_1 \rangle \,=\,Tr\left(P^N\right)\,=\,\lambda_1^N\,+\,\lambda_2^N,
    \label{eq: part_func_simp}
\end{equation}

dove $\lambda_1$ e $\lambda_2$ sono gli autovalori della matrice P. La loro determinazione richiede la soluzione di un problema agli 
autovalori, che porta a 

\begin{equation}
    \lambda_{1,2}\,=\,e^{\beta J} \cosh{\left(\beta h\right)}\,\pm\,\sqrt{e^{- 2 \beta J}\,+\,e^{2 \beta J} \sinh^2{\left(\beta h\right)}}.
    \label{eq: autoval_P}
\end{equation}

Una ottima approssimazione, quando il numero di spin preso in considerazione è elevato, consiste nel trascurare il secondo autovalore 
dato che 

\begin{equation}
    \lim_{N \to \infty} \left(\frac{\lambda_2}{\lambda_1}\right)^N\,=\,0.
    \label{eq: approx_Q}
\end{equation}

L'energia libera, dalla quale è possibile determinare tutta la termodinamica del sistema, risulta quindi

\begin{equation}
    A\left(h,\,T\right)\,=\,-k_B T \ln{\left[Q\left(h,\,T\right)\right]}\,\simeq\,-Nk_BT \ln{\left(\lambda_1\right)}.
    \label{eq: en_lib}
\end{equation}


\subsection{Teoria di Campo Medio}
\newpage



\end{document}






%% Struttura delle sezioni
%\section{Argomento Principale 1}
%In questa sezione vengono discusse le basi teoriche dell'argomento principale.
%
%\subsection{Definizioni e Teoremi}
%Ecco un esempio di definizione e di teorema:
%
%\begin{definition}[Esempio di Definizione]
%Una definizione descrive un concetto fondamentale che verrà utilizzato successivamente.
%\end{definition}
%
%\begin{theorem}[Esempio di Teorema]
%Se $a = b$, allora $a^2 = b^2$.
%\end{theorem}
%
%\begin{proof}
%La dimostrazione segue direttamente dalla proprietà di uguaglianza. Molto semplice!
%\end{proof}
%
%\subsection{Esempi}
%Esempio pratico:
%\begin{itemize}
%    \item Se $a = 3$ e $b = 3$, allora $a^2 = b^2 = 9$.
%\end{itemize}
%
%\section{Argomento Principale 2}
%Secondo argomento con approfondimenti. Può includere grafici, equazioni e riferimenti incrociati.
%
%\begin{figure}[h!]
%    \centering
%    \includegraphics[width=0.6\textwidth]{example-image} % Immagine esempio
%    \caption{Esempio di figura.}
%    \label{fig:example}
%\end{figure}
%
%Come mostrato in \Cref{fig:example}, il grafico rappresenta un esempio pratico.
%
%\section{Conclusioni}
%Breve riassunto dei risultati principali e considerazioni finali.
%
%% Bibliografia
%\begin{thebibliography}{9}
%    \bibitem{lamport94} Leslie Lamport, \textit{LaTeX: A Document Preparation System}, Addison Wesley, 1994.
%    \bibitem{knuth84} Donald Knuth, \textit{The TeXbook}, Addison-Wesley, 1984.
%\end{thebibliography}
%
